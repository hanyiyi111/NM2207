% Options for packages loaded elsewhere
\PassOptionsToPackage{unicode}{hyperref}
\PassOptionsToPackage{hyphens}{url}
%
\documentclass[
]{article}
\usepackage{amsmath,amssymb}
\usepackage{iftex}
\ifPDFTeX
  \usepackage[T1]{fontenc}
  \usepackage[utf8]{inputenc}
  \usepackage{textcomp} % provide euro and other symbols
\else % if luatex or xetex
  \usepackage{unicode-math} % this also loads fontspec
  \defaultfontfeatures{Scale=MatchLowercase}
  \defaultfontfeatures[\rmfamily]{Ligatures=TeX,Scale=1}
\fi
\usepackage{lmodern}
\ifPDFTeX\else
  % xetex/luatex font selection
\fi
% Use upquote if available, for straight quotes in verbatim environments
\IfFileExists{upquote.sty}{\usepackage{upquote}}{}
\IfFileExists{microtype.sty}{% use microtype if available
  \usepackage[]{microtype}
  \UseMicrotypeSet[protrusion]{basicmath} % disable protrusion for tt fonts
}{}
\makeatletter
\@ifundefined{KOMAClassName}{% if non-KOMA class
  \IfFileExists{parskip.sty}{%
    \usepackage{parskip}
  }{% else
    \setlength{\parindent}{0pt}
    \setlength{\parskip}{6pt plus 2pt minus 1pt}}
}{% if KOMA class
  \KOMAoptions{parskip=half}}
\makeatother
\usepackage{xcolor}
\usepackage[margin=1in]{geometry}
\usepackage{color}
\usepackage{fancyvrb}
\newcommand{\VerbBar}{|}
\newcommand{\VERB}{\Verb[commandchars=\\\{\}]}
\DefineVerbatimEnvironment{Highlighting}{Verbatim}{commandchars=\\\{\}}
% Add ',fontsize=\small' for more characters per line
\usepackage{framed}
\definecolor{shadecolor}{RGB}{248,248,248}
\newenvironment{Shaded}{\begin{snugshade}}{\end{snugshade}}
\newcommand{\AlertTok}[1]{\textcolor[rgb]{0.94,0.16,0.16}{#1}}
\newcommand{\AnnotationTok}[1]{\textcolor[rgb]{0.56,0.35,0.01}{\textbf{\textit{#1}}}}
\newcommand{\AttributeTok}[1]{\textcolor[rgb]{0.13,0.29,0.53}{#1}}
\newcommand{\BaseNTok}[1]{\textcolor[rgb]{0.00,0.00,0.81}{#1}}
\newcommand{\BuiltInTok}[1]{#1}
\newcommand{\CharTok}[1]{\textcolor[rgb]{0.31,0.60,0.02}{#1}}
\newcommand{\CommentTok}[1]{\textcolor[rgb]{0.56,0.35,0.01}{\textit{#1}}}
\newcommand{\CommentVarTok}[1]{\textcolor[rgb]{0.56,0.35,0.01}{\textbf{\textit{#1}}}}
\newcommand{\ConstantTok}[1]{\textcolor[rgb]{0.56,0.35,0.01}{#1}}
\newcommand{\ControlFlowTok}[1]{\textcolor[rgb]{0.13,0.29,0.53}{\textbf{#1}}}
\newcommand{\DataTypeTok}[1]{\textcolor[rgb]{0.13,0.29,0.53}{#1}}
\newcommand{\DecValTok}[1]{\textcolor[rgb]{0.00,0.00,0.81}{#1}}
\newcommand{\DocumentationTok}[1]{\textcolor[rgb]{0.56,0.35,0.01}{\textbf{\textit{#1}}}}
\newcommand{\ErrorTok}[1]{\textcolor[rgb]{0.64,0.00,0.00}{\textbf{#1}}}
\newcommand{\ExtensionTok}[1]{#1}
\newcommand{\FloatTok}[1]{\textcolor[rgb]{0.00,0.00,0.81}{#1}}
\newcommand{\FunctionTok}[1]{\textcolor[rgb]{0.13,0.29,0.53}{\textbf{#1}}}
\newcommand{\ImportTok}[1]{#1}
\newcommand{\InformationTok}[1]{\textcolor[rgb]{0.56,0.35,0.01}{\textbf{\textit{#1}}}}
\newcommand{\KeywordTok}[1]{\textcolor[rgb]{0.13,0.29,0.53}{\textbf{#1}}}
\newcommand{\NormalTok}[1]{#1}
\newcommand{\OperatorTok}[1]{\textcolor[rgb]{0.81,0.36,0.00}{\textbf{#1}}}
\newcommand{\OtherTok}[1]{\textcolor[rgb]{0.56,0.35,0.01}{#1}}
\newcommand{\PreprocessorTok}[1]{\textcolor[rgb]{0.56,0.35,0.01}{\textit{#1}}}
\newcommand{\RegionMarkerTok}[1]{#1}
\newcommand{\SpecialCharTok}[1]{\textcolor[rgb]{0.81,0.36,0.00}{\textbf{#1}}}
\newcommand{\SpecialStringTok}[1]{\textcolor[rgb]{0.31,0.60,0.02}{#1}}
\newcommand{\StringTok}[1]{\textcolor[rgb]{0.31,0.60,0.02}{#1}}
\newcommand{\VariableTok}[1]{\textcolor[rgb]{0.00,0.00,0.00}{#1}}
\newcommand{\VerbatimStringTok}[1]{\textcolor[rgb]{0.31,0.60,0.02}{#1}}
\newcommand{\WarningTok}[1]{\textcolor[rgb]{0.56,0.35,0.01}{\textbf{\textit{#1}}}}
\usepackage{graphicx}
\makeatletter
\def\maxwidth{\ifdim\Gin@nat@width>\linewidth\linewidth\else\Gin@nat@width\fi}
\def\maxheight{\ifdim\Gin@nat@height>\textheight\textheight\else\Gin@nat@height\fi}
\makeatother
% Scale images if necessary, so that they will not overflow the page
% margins by default, and it is still possible to overwrite the defaults
% using explicit options in \includegraphics[width, height, ...]{}
\setkeys{Gin}{width=\maxwidth,height=\maxheight,keepaspectratio}
% Set default figure placement to htbp
\makeatletter
\def\fps@figure{htbp}
\makeatother
\setlength{\emergencystretch}{3em} % prevent overfull lines
\providecommand{\tightlist}{%
  \setlength{\itemsep}{0pt}\setlength{\parskip}{0pt}}
\setcounter{secnumdepth}{-\maxdimen} % remove section numbering
\ifLuaTeX
  \usepackage{selnolig}  % disable illegal ligatures
\fi
\IfFileExists{bookmark.sty}{\usepackage{bookmark}}{\usepackage{hyperref}}
\IfFileExists{xurl.sty}{\usepackage{xurl}}{} % add URL line breaks if available
\urlstyle{same}
\hypersetup{
  pdftitle={Challenge-3},
  pdfauthor={HAN YIYI},
  hidelinks,
  pdfcreator={LaTeX via pandoc}}

\title{Challenge-3}
\author{HAN YIYI}
\date{30/8/2023}

\begin{document}
\maketitle

\hypertarget{i.-questions}{%
\subsection{I. Questions}\label{i.-questions}}

\hypertarget{question-1-emoji-expressions}{%
\paragraph{Question 1: Emoji
Expressions}\label{question-1-emoji-expressions}}

Imagine you're analyzing social media posts for sentiment analysis. If
you were to create a variable named ``postSentiment'' to store the
sentiment of a post using emojis (😄 for positive, 😐 for neutral, 😢
for negative), what data type would you assign to this variable? Why?
(\emph{narrative type question, no code required})

\textbf{Solution:} \emph{Character (string of alphabets). Because this
is a categoric variable and it is ordinal. The three sentiments have
natural ordering.}

\hypertarget{question-2-hashtag-havoc}{%
\paragraph{Question 2: Hashtag Havoc}\label{question-2-hashtag-havoc}}

In a study on trending hashtags, you want to store the list of hashtags
associated with a post. What data type would you choose for the variable
``postHashtags''? How might this data type help you analyze and
categorize the hashtags later? (\emph{narrative type question, no code
required})

\textbf{Solution:} \emph{Character (string of alphabets). We can get the
number of selected hashtags by counting how many entries they have. }

\hypertarget{question-3-time-travelers-log}{%
\paragraph{Question 3: Time Traveler's
Log}\label{question-3-time-travelers-log}}

You're examining the timing of user interactions on a website. Would you
use a numeric or non-numeric data type to represent the timestamp of
each interaction? Explain your choice (\emph{narrative type question, no
code required})

\textbf{Solution:} \emph{I would use a numeric data type because we can
use numbers to indicate time.}

\hypertarget{question-4-event-elegance}{%
\paragraph{Question 4: Event Elegance}\label{question-4-event-elegance}}

You're managing an event database that includes the date and time of
each session. What data type(s) would you use to represent the session
date and time? (\emph{narrative type question, no code required})

\textbf{Solution:} \emph{Character (string of numbers)}

\hypertarget{question-5-nominee-nominations}{%
\paragraph{Question 5: Nominee
Nominations}\label{question-5-nominee-nominations}}

You're analyzing nominations for an online award. Each participant can
nominate multiple candidates. What data type would be suitable for
storing the list of nominated candidates for each participant?
(\emph{narrative type question, no code required})

\textbf{Solution:} \emph{Character (string of alphabets)}

\hypertarget{question-6-communication-channels}{%
\paragraph{Question 6: Communication
Channels}\label{question-6-communication-channels}}

In a survey about preferred communication channels, respondents choose
from options like ``email,'' ``phone,'' or ``social media.'' What data
type would you assign to the variable ``preferredChannel''?
(\emph{narrative type question, no code required})

\textbf{Solution:} \emph{Character (string of alphabets)}

\hypertarget{question-7-colorful-commentary}{%
\paragraph{Question 7: Colorful
Commentary}\label{question-7-colorful-commentary}}

In a design feedback survey, participants are asked to describe their
feelings about a website using color names (e.g., ``warm red,'' ``cool
blue''). What data type would you choose for the variable
``feedbackColor''? (\emph{narrative type question, no code required})

\textbf{Solution:} \emph{Character (string of alphabets)}

\hypertarget{question-8-variable-exploration}{%
\paragraph{Question 8: Variable
Exploration}\label{question-8-variable-exploration}}

Imagine you're conducting a study on social media usage. Identify three
variables related to this study, and specify their data types in R.
Classify each variable as either numeric or non-numeric.

\textbf{Solution:} \emph{1.''PreferredPlatform'',non-numeric,character;
2.''Gender'',non-numeric,character;
3.''DailyHoursOfUsage'',numeric,double.}

\hypertarget{question-9-vector-variety}{%
\paragraph{Question 9: Vector Variety}\label{question-9-vector-variety}}

Create a numeric vector named ``ages'' containing the ages of five
people: 25, 30, 22, 28, and 33. Print the vector.

\textbf{Solution:}

\begin{Shaded}
\begin{Highlighting}[]
\CommentTok{\# Enter code here}
\NormalTok{ages }\OtherTok{\textless{}{-}} \FunctionTok{c}\NormalTok{(}\DecValTok{25}\NormalTok{,}\DecValTok{30}\NormalTok{,}\DecValTok{22}\NormalTok{,}\DecValTok{28}\NormalTok{,}\DecValTok{33}\NormalTok{)}
\FunctionTok{print}\NormalTok{(ages)}
\end{Highlighting}
\end{Shaded}

\begin{verbatim}
## [1] 25 30 22 28 33
\end{verbatim}

\hypertarget{question-10-list-logic}{%
\paragraph{Question 10: List Logic}\label{question-10-list-logic}}

Construct a list named ``student\_info'' that contains the following
elements:

\begin{itemize}
\item
  A character vector of student names: ``Alice,'' ``Bob,'' ``Catherine''
\item
  A numeric vector of their respective scores: 85, 92, 78
\item
  A logical vector indicating if they passed the exam: TRUE, TRUE, FALSE
\end{itemize}

Print the list.

\textbf{Solution:}

\begin{Shaded}
\begin{Highlighting}[]
\CommentTok{\# Enter code here}
\NormalTok{student\_info }\OtherTok{=} \FunctionTok{list}\NormalTok{(}\AttributeTok{name=}\FunctionTok{c}\NormalTok{(}\StringTok{"Alice"}\NormalTok{,}\StringTok{"Bob"}\NormalTok{,}\StringTok{"Catherine"}\NormalTok{), }\AttributeTok{score=}\FunctionTok{c}\NormalTok{(}\DecValTok{85}\NormalTok{,}\DecValTok{92}\NormalTok{,}\DecValTok{78}\NormalTok{), }\AttributeTok{passed.the.exam=}\FunctionTok{c}\NormalTok{(}\ConstantTok{TRUE}\NormalTok{,}\ConstantTok{TRUE}\NormalTok{,}\ConstantTok{FALSE}\NormalTok{))}
\FunctionTok{print}\NormalTok{(student\_info)}
\end{Highlighting}
\end{Shaded}

\begin{verbatim}
## $name
## [1] "Alice"     "Bob"       "Catherine"
## 
## $score
## [1] 85 92 78
## 
## $passed.the.exam
## [1]  TRUE  TRUE FALSE
\end{verbatim}

\hypertarget{question-11-type-tracking}{%
\paragraph{Question 11: Type Tracking}\label{question-11-type-tracking}}

You have a vector ``data'' containing the values 10, 15.5, ``20'', and
TRUE. Determine the data types of each element using the typeof()
function.

\textbf{Solution:}

\begin{Shaded}
\begin{Highlighting}[]
\CommentTok{\# Enter code here}
\NormalTok{a }\OtherTok{\textless{}{-}} \FunctionTok{c}\NormalTok{(}\DecValTok{10}\NormalTok{)}
\FunctionTok{typeof}\NormalTok{(a)}
\end{Highlighting}
\end{Shaded}

\begin{verbatim}
## [1] "double"
\end{verbatim}

\begin{Shaded}
\begin{Highlighting}[]
\NormalTok{b }\OtherTok{\textless{}{-}} \FunctionTok{c}\NormalTok{(}\FloatTok{15.5}\NormalTok{)}
\FunctionTok{typeof}\NormalTok{(b)}
\end{Highlighting}
\end{Shaded}

\begin{verbatim}
## [1] "double"
\end{verbatim}

\begin{Shaded}
\begin{Highlighting}[]
\NormalTok{c }\OtherTok{\textless{}{-}} \FunctionTok{c}\NormalTok{(}\StringTok{"20"}\NormalTok{)}
\FunctionTok{typeof}\NormalTok{(c)}
\end{Highlighting}
\end{Shaded}

\begin{verbatim}
## [1] "character"
\end{verbatim}

\begin{Shaded}
\begin{Highlighting}[]
\NormalTok{d }\OtherTok{\textless{}{-}} \FunctionTok{c}\NormalTok{(}\ConstantTok{TRUE}\NormalTok{)}
\FunctionTok{typeof}\NormalTok{(d)}
\end{Highlighting}
\end{Shaded}

\begin{verbatim}
## [1] "logical"
\end{verbatim}

\hypertarget{question-12-coercion-chronicles}{%
\paragraph{Question 12: Coercion
Chronicles}\label{question-12-coercion-chronicles}}

You have a numeric vector ``prices'' with values 20.5, 15, and ``25''.
Use explicit coercion to convert the last element to a numeric data
type. Print the updated vector.

\textbf{Solution:}

\begin{Shaded}
\begin{Highlighting}[]
\CommentTok{\# Enter code here}
\NormalTok{prices }\OtherTok{\textless{}{-}} \FunctionTok{c}\NormalTok{(}\FloatTok{20.5}\NormalTok{,}\DecValTok{15}\NormalTok{,}\StringTok{"25"}\NormalTok{)}
\NormalTok{prices }\OtherTok{\textless{}{-}} \FunctionTok{as.double}\NormalTok{(prices)}
\FunctionTok{typeof}\NormalTok{(prices)}
\end{Highlighting}
\end{Shaded}

\begin{verbatim}
## [1] "double"
\end{verbatim}

\begin{Shaded}
\begin{Highlighting}[]
\FunctionTok{print}\NormalTok{(prices)}
\end{Highlighting}
\end{Shaded}

\begin{verbatim}
## [1] 20.5 15.0 25.0
\end{verbatim}

\hypertarget{question-13-implicit-intuition}{%
\paragraph{Question 13: Implicit
Intuition}\label{question-13-implicit-intuition}}

Combine the numeric vector c(5, 10, 15) with the character vector
c(``apple'', ``banana'', ``cherry''). What happens to the data types of
the combined vector? Explain the concept of implicit coercion.

\textbf{Solution:} \emph{When different types of vectors are combined, R
will automatically convert the vector to accommodate all element types
based on the content.}

\begin{Shaded}
\begin{Highlighting}[]
\CommentTok{\# Enter code here}
\NormalTok{x }\OtherTok{\textless{}{-}} \FunctionTok{c}\NormalTok{(}\DecValTok{5}\NormalTok{,}\DecValTok{10}\NormalTok{,}\DecValTok{15}\NormalTok{)}
\NormalTok{x }\OtherTok{\textless{}{-}} \FunctionTok{c}\NormalTok{(x,}\StringTok{"apple"}\NormalTok{,}\StringTok{"banana"}\NormalTok{,}\StringTok{"cherry"}\NormalTok{)}
\FunctionTok{typeof}\NormalTok{(x)}
\end{Highlighting}
\end{Shaded}

\begin{verbatim}
## [1] "character"
\end{verbatim}

\hypertarget{question-14-coercion-challenges}{%
\paragraph{Question 14: Coercion
Challenges}\label{question-14-coercion-challenges}}

You have a vector ``numbers'' with values 7, 12.5, and ``15.7''.
Calculate the sum of these numbers. Will R automatically handle the data
type conversion? If not, how would you handle it?

\textbf{Solution:} R will turn convert double vector into character. I
would explicitly coerce the vector to be double.

\begin{Shaded}
\begin{Highlighting}[]
\CommentTok{\# Enter code here}
\NormalTok{numbers }\OtherTok{\textless{}{-}} \FunctionTok{c}\NormalTok{(}\DecValTok{7}\NormalTok{,}\FloatTok{12.5}\NormalTok{)}
\NormalTok{numbers }\OtherTok{\textless{}{-}} \FunctionTok{c}\NormalTok{(numbers,}\StringTok{"15.7"}\NormalTok{)}
\FunctionTok{typeof}\NormalTok{(numbers)}
\end{Highlighting}
\end{Shaded}

\begin{verbatim}
## [1] "character"
\end{verbatim}

\begin{Shaded}
\begin{Highlighting}[]
\NormalTok{numbers }\OtherTok{\textless{}{-}} \FunctionTok{as.double}\NormalTok{(numbers)}
\FunctionTok{typeof}\NormalTok{(numbers)}
\end{Highlighting}
\end{Shaded}

\begin{verbatim}
## [1] "double"
\end{verbatim}

\begin{Shaded}
\begin{Highlighting}[]
\FunctionTok{sum}\NormalTok{(numbers)}
\end{Highlighting}
\end{Shaded}

\begin{verbatim}
## [1] 35.2
\end{verbatim}

\hypertarget{question-15-coercion-consequences}{%
\paragraph{Question 15: Coercion
Consequences}\label{question-15-coercion-consequences}}

Suppose you want to calculate the average of a vector ``grades'' with
values 85, 90.5, and ``75.2''. If you directly calculate the mean using
the mean() function, what result do you expect? How might you ensure
accurate calculation?

\textbf{Solution:}

\begin{Shaded}
\begin{Highlighting}[]
\CommentTok{\# Enter code here}
\NormalTok{grades }\OtherTok{\textless{}{-}} \FunctionTok{c}\NormalTok{(}\DecValTok{85}\NormalTok{,}\FloatTok{90.5}\NormalTok{)}
\NormalTok{grades }\OtherTok{\textless{}{-}} \FunctionTok{c}\NormalTok{(grades,}\StringTok{"75.2"}\NormalTok{)}
\NormalTok{grades }\OtherTok{\textless{}{-}} \FunctionTok{as.double}\NormalTok{(grades)}
\FunctionTok{mean}\NormalTok{(grades)}
\end{Highlighting}
\end{Shaded}

\begin{verbatim}
## [1] 83.56667
\end{verbatim}

\hypertarget{question-16-data-diversity-in-lists}{%
\paragraph{Question 16: Data Diversity in
Lists}\label{question-16-data-diversity-in-lists}}

Create a list named ``mixed\_data'' with the following components:

\begin{itemize}
\item
  A numeric vector: 10, 20, 30
\item
  A character vector: ``red'', ``green'', ``blue''
\item
  A logical vector: TRUE, FALSE, TRUE
\end{itemize}

Calculate the mean of the numeric vector within the list.

\textbf{Solution:}

\begin{Shaded}
\begin{Highlighting}[]
\CommentTok{\# Enter code here}
\NormalTok{mixed\_data }\OtherTok{=} \FunctionTok{list}\NormalTok{(}\AttributeTok{numeric=}\FunctionTok{c}\NormalTok{(}\DecValTok{10}\NormalTok{,}\DecValTok{20}\NormalTok{,}\DecValTok{30}\NormalTok{), }\AttributeTok{character=}\FunctionTok{c}\NormalTok{(}\StringTok{"red"}\NormalTok{,}\StringTok{"green"}\NormalTok{,}\StringTok{"blue"}\NormalTok{), }\AttributeTok{logical=}\FunctionTok{c}\NormalTok{(}\ConstantTok{TRUE}\NormalTok{,}\ConstantTok{FALSE}\NormalTok{,}\ConstantTok{TRUE}\NormalTok{))}
\NormalTok{mixed\_data}\SpecialCharTok{$}\NormalTok{numeric}
\end{Highlighting}
\end{Shaded}

\begin{verbatim}
## [1] 10 20 30
\end{verbatim}

\begin{Shaded}
\begin{Highlighting}[]
\FunctionTok{mean}\NormalTok{(mixed\_data}\SpecialCharTok{$}\NormalTok{numeric)}
\end{Highlighting}
\end{Shaded}

\begin{verbatim}
## [1] 20
\end{verbatim}

\hypertarget{question-17-list-logic-follow-up}{%
\paragraph{Question 17: List Logic
Follow-up}\label{question-17-list-logic-follow-up}}

Using the ``student\_info'' list from Question 10, extract and print the
score of the student named ``Bob.''

\textbf{Solution:}

\begin{Shaded}
\begin{Highlighting}[]
\CommentTok{\# Enter code here}
\NormalTok{student\_info}\SpecialCharTok{$}\NormalTok{score[student\_info}\SpecialCharTok{$}\NormalTok{name}\SpecialCharTok{==}\StringTok{"Bob"}\NormalTok{]  }\CommentTok{\# \textless{}{-}{-} pipe operator}
\end{Highlighting}
\end{Shaded}

\begin{verbatim}
## [1] 92
\end{verbatim}

\hypertarget{question-18-dynamic-access}{%
\paragraph{Question 18: Dynamic
Access}\label{question-18-dynamic-access}}

Create a numeric vector values with random values. Write R code to
dynamically access and print the last element of the vector, regardless
of its length.

\textbf{Solution:}

\begin{Shaded}
\begin{Highlighting}[]
\CommentTok{\# Enter code here}
\NormalTok{x}\OtherTok{\textless{}{-}}\FunctionTok{c}\NormalTok{(}\DecValTok{6}\NormalTok{,}\DecValTok{7}\NormalTok{,}\DecValTok{8}\NormalTok{,}\DecValTok{2}\NormalTok{,}\DecValTok{6}\NormalTok{)}
\NormalTok{x[}\FunctionTok{length}\NormalTok{(x)]}
\end{Highlighting}
\end{Shaded}

\begin{verbatim}
## [1] 6
\end{verbatim}

\hypertarget{question-19-multiple-matches}{%
\paragraph{Question 19: Multiple
Matches}\label{question-19-multiple-matches}}

You have a character vector words \textless- c(``apple'', ``banana'',
``cherry'', ``apple''). Write R code to find and print the indices of
all occurrences of the word ``apple.''

\textbf{Solution:}

\begin{Shaded}
\begin{Highlighting}[]
\CommentTok{\# Enter code here}
\NormalTok{words }\OtherTok{\textless{}{-}} \FunctionTok{c}\NormalTok{(}\StringTok{"apple"}\NormalTok{,}\StringTok{"banana"}\NormalTok{,}\StringTok{"cherry"}\NormalTok{,}\StringTok{"apple"}\NormalTok{)}
\NormalTok{words }\SpecialCharTok{==} \StringTok{"apple"}
\end{Highlighting}
\end{Shaded}

\begin{verbatim}
## [1]  TRUE FALSE FALSE  TRUE
\end{verbatim}

\begin{Shaded}
\begin{Highlighting}[]
\FunctionTok{which}\NormalTok{(words }\SpecialCharTok{==} \StringTok{"apple"}\NormalTok{)}
\end{Highlighting}
\end{Shaded}

\begin{verbatim}
## [1] 1 4
\end{verbatim}

\hypertarget{question-20-conditional-capture}{%
\paragraph{Question 20: Conditional
Capture}\label{question-20-conditional-capture}}

Assume you have a vector ages containing the ages of individuals. Write
R code to extract and print the ages of individuals who are older than
30.

\textbf{Solution:}

\begin{Shaded}
\begin{Highlighting}[]
\CommentTok{\# Enter code here}

\NormalTok{ages[ages}\SpecialCharTok{\textgreater{}}\DecValTok{30}\NormalTok{]}
\end{Highlighting}
\end{Shaded}

\begin{verbatim}
## [1] 33
\end{verbatim}

\hypertarget{question-21-extract-every-nth}{%
\paragraph{Question 21: Extract Every
Nth}\label{question-21-extract-every-nth}}

Given a numeric vector sequence \textless- 1:20, write R code to extract
and print every third element of the vector.

\textbf{Solution:}

\begin{Shaded}
\begin{Highlighting}[]
\CommentTok{\# Enter code here}
\NormalTok{x}\OtherTok{\textless{}{-}}\FunctionTok{seq}\NormalTok{(}\AttributeTok{from=}\DecValTok{1}\NormalTok{,}\AttributeTok{to=}\DecValTok{20}\NormalTok{,}\AttributeTok{by=}\DecValTok{3}\NormalTok{)}
\FunctionTok{print}\NormalTok{(x)}
\end{Highlighting}
\end{Shaded}

\begin{verbatim}
## [1]  1  4  7 10 13 16 19
\end{verbatim}

\hypertarget{question-22-range-retrieval}{%
\paragraph{Question 22: Range
Retrieval}\label{question-22-range-retrieval}}

Create a numeric vector numbers with values from 1 to 10. Write R code
to extract and print the values between the fourth and eighth elements.

\textbf{Solution:}

\begin{Shaded}
\begin{Highlighting}[]
\CommentTok{\# Enter code here}
\NormalTok{x }\OtherTok{\textless{}{-}} \FunctionTok{c}\NormalTok{(}\DecValTok{1}\NormalTok{,}\DecValTok{2}\NormalTok{,}\DecValTok{3}\NormalTok{,}\DecValTok{4}\NormalTok{,}\DecValTok{5}\NormalTok{,}\DecValTok{6}\NormalTok{,}\DecValTok{7}\NormalTok{,}\DecValTok{8}\NormalTok{,}\DecValTok{9}\NormalTok{,}\DecValTok{10}\NormalTok{)}
\NormalTok{x[}\FunctionTok{c}\NormalTok{(}\DecValTok{4}\NormalTok{,}\DecValTok{8}\NormalTok{)]}
\end{Highlighting}
\end{Shaded}

\begin{verbatim}
## [1] 4 8
\end{verbatim}

\hypertarget{question-23-missing-matters}{%
\paragraph{Question 23: Missing
Matters}\label{question-23-missing-matters}}

Suppose you have a numeric vector data \textless- c(10, NA, 15, 20).
Write R code to check if the second element of the vector is missing
(NA).

\textbf{Solution:}

\begin{Shaded}
\begin{Highlighting}[]
\CommentTok{\# Enter code here}
\NormalTok{x}\OtherTok{\textless{}{-}}\FunctionTok{c}\NormalTok{(}\DecValTok{10}\NormalTok{,}\ConstantTok{NA}\NormalTok{,}\DecValTok{15}\NormalTok{,}\DecValTok{20}\NormalTok{)}
\NormalTok{x[}\FunctionTok{c}\NormalTok{(}\DecValTok{2}\NormalTok{)]}
\end{Highlighting}
\end{Shaded}

\begin{verbatim}
## [1] NA
\end{verbatim}

\hypertarget{question-24-temperature-extremes}{%
\paragraph{Question 24: Temperature
Extremes}\label{question-24-temperature-extremes}}

Assume you have a numeric vector temperatures with daily temperatures.
Create a logical vector hot\_days that flags days with temperatures
above 90 degrees Fahrenheit. Print the total number of hot days.

\textbf{Solution:}

\begin{Shaded}
\begin{Highlighting}[]
\CommentTok{\# Enter code here}
\NormalTok{temperatures}\OtherTok{\textless{}{-}}\FunctionTok{c}\NormalTok{(}\DecValTok{85}\NormalTok{,}\DecValTok{88}\NormalTok{,}\DecValTok{91}\NormalTok{,}\DecValTok{93}\NormalTok{,}\DecValTok{79}\NormalTok{,}\DecValTok{98}\NormalTok{,}\DecValTok{85}\NormalTok{,}\DecValTok{92}\NormalTok{)}
\NormalTok{hot\_days}\OtherTok{\textless{}{-}}\NormalTok{temperatures}\SpecialCharTok{\textgreater{}}\DecValTok{90}
\NormalTok{hot\_days}
\end{Highlighting}
\end{Shaded}

\begin{verbatim}
## [1] FALSE FALSE  TRUE  TRUE FALSE  TRUE FALSE  TRUE
\end{verbatim}

\begin{Shaded}
\begin{Highlighting}[]
\FunctionTok{sum}\NormalTok{(hot\_days)}
\end{Highlighting}
\end{Shaded}

\begin{verbatim}
## [1] 4
\end{verbatim}

\hypertarget{question-25-string-selection}{%
\paragraph{Question 25: String
Selection}\label{question-25-string-selection}}

Given a character vector fruits containing fruit names, create a logical
vector long\_names that identifies fruits with names longer than 6
characters. Print the long fruit names.

\textbf{Solution:}

\begin{Shaded}
\begin{Highlighting}[]
\CommentTok{\# Enter code here}
\NormalTok{fruits }\OtherTok{\textless{}{-}}\FunctionTok{c}\NormalTok{(}\StringTok{"apple"}\NormalTok{,}\StringTok{"strawberry"}\NormalTok{,}\StringTok{"blueberry"}\NormalTok{,}\StringTok{"banana"}\NormalTok{)}
\NormalTok{long\_fruit\_names}\OtherTok{\textless{}{-}}\NormalTok{fruits[}\FunctionTok{nchar}\NormalTok{(fruits)}\SpecialCharTok{\textgreater{}}\DecValTok{6}\NormalTok{]}
\FunctionTok{print}\NormalTok{(long\_fruit\_names)}
\end{Highlighting}
\end{Shaded}

\begin{verbatim}
## [1] "strawberry" "blueberry"
\end{verbatim}

\hypertarget{question-26-data-divisibility}{%
\paragraph{Question 26: Data
Divisibility}\label{question-26-data-divisibility}}

Given a numeric vector numbers, create a logical vector divisible\_by\_5
to indicate numbers that are divisible by 5. Print the numbers that
satisfy this condition.

\textbf{Solution:}

\begin{Shaded}
\begin{Highlighting}[]
\CommentTok{\# Enter code here}
\NormalTok{x }\OtherTok{\textless{}{-}}\FunctionTok{c}\NormalTok{(}\DecValTok{10}\NormalTok{,}\DecValTok{8}\NormalTok{,}\DecValTok{11}\NormalTok{)}
\NormalTok{divisible\_by\_5}\OtherTok{\textless{}{-}}\NormalTok{x[(x }\SpecialCharTok{\%\%} \DecValTok{5}\NormalTok{)}\SpecialCharTok{==}\DecValTok{0}\NormalTok{]}
\FunctionTok{print}\NormalTok{(divisible\_by\_5)}
\end{Highlighting}
\end{Shaded}

\begin{verbatim}
## [1] 10
\end{verbatim}

\hypertarget{question-27-bigger-or-smaller}{%
\paragraph{Question 27: Bigger or
Smaller?}\label{question-27-bigger-or-smaller}}

You have two numeric vectors vector1 and vector2. Create a logical
vector comparison to indicate whether each element in vector1 is greater
than the corresponding element in vector2. Print the comparison results.

\textbf{Solution:}

\begin{Shaded}
\begin{Highlighting}[]
\CommentTok{\# Enter code here}
\NormalTok{x }\OtherTok{\textless{}{-}}\FunctionTok{c}\NormalTok{(}\DecValTok{3}\NormalTok{,}\DecValTok{3}\NormalTok{,}\DecValTok{5}\NormalTok{,}\DecValTok{6}\NormalTok{,}\DecValTok{7}\NormalTok{,}\DecValTok{8}\NormalTok{)}
\NormalTok{y }\OtherTok{\textless{}{-}}\FunctionTok{c}\NormalTok{(}\DecValTok{5}\NormalTok{,}\DecValTok{6}\NormalTok{,}\DecValTok{3}\NormalTok{,}\DecValTok{1}\NormalTok{,}\DecValTok{3}\NormalTok{,}\DecValTok{5}\NormalTok{)}
\NormalTok{x}\SpecialCharTok{\textgreater{}}\NormalTok{y}
\end{Highlighting}
\end{Shaded}

\begin{verbatim}
## [1] FALSE FALSE  TRUE  TRUE  TRUE  TRUE
\end{verbatim}

\end{document}
